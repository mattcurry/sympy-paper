\documentclass[answers,12pt]{exam}
\usepackage{xcolor}
\usepackage{hyperref}

\unframedsolutions
\shadedsolutions
\renewcommand{\solutiontitle}{}

\begin{document}

Dear Editors, \bigskip

Please read below our answer about the
questions made by you and reviewers. I hope that you agree
with all our comments. \bigskip

Best regards,\bigskip

The authors

\pagebreak

\section{Comments from Editor}

\begin{questions}
\question I found the \url{http://live.sympy.org/} site very convenient for trying things
out in SymPy, but it is only mentioned in the supplement, which might be
overlooked by readers. I suggest moving that section into the main paper, as
it provides a great way to play along with the examples while reading the
paper.
\begin{solution}
  % TODO
\end{solution}
\question The Basic Usage section omits what I think is an important point:
how does one distinguish between evaluating exp(1) in Python and exp(1) in
SymPy? In other words, how are symbolic constants specified?
\begin{solution}
  % TODO. I'm unclear what the editor is asking here.
\end{solution}
\question The first
thing I tried after seeing Table 2 (simplication functions) did not work:

\begin{verbatim}
>>> trigsimp (exp( Matrix(2, 2, [0, -y, y, 0])) )
\end{verbatim}
fails to recognize cos and
sin. Am I expecting too much?
\begin{solution}
  This indeed does not work. The implementation in \verb|trigsimp| primarily
  thus far has focused on transforming trigonometric functions into other
  trigonometric functions. The ability to simplify complex exponentials is
  something that we would like to work.
  \href{https://github.com/sympy/sympy/issues/11459}{Issue 11459} in our
  public issue tracker tracks this feature.
\end{solution}
\question Section 3.6: are eigenvalues and singular
values included? If not, why not?
\begin{solution}
  % TODO
\end{solution}
\question Like one of the referees, I expected to
see MPFR mentioned in Section 4.1, at least to mention the pros and cons and
say why it isn’t used.
\begin{solution}
  % TODO
\end{solution}
\question Section 4.1: please state whether the syntax for
functions in mpmath is identical, or not, to that in Sympy.
\begin{solution}
  % TODO
\end{solution}
\question Unless I have
missed something, a weakness of the paper is that speed is mentioned only
twice, on lines 496 and 666. My assumption is that SymPy is slow compared with
its commercial competitors. Please comment on the speed issue, and not just in
the conclusions.
\begin{solution}
  % TODO
\end{solution}
\question Editors are needed for [20].
\begin{solution}
  % TODO: I do not know how to make bibtex (with apalike) show the editor
  % here.
\end{solution}
\end{questions}

\section{Reviewer 1}

\begin{questions}
\question one
\begin{solution}
 We opted to follow the ABC style of ...
\end{solution}
\question  two
\begin{solution}
Section 2 has been updated and the discussion ...
\end{solution}
\end{questions}

\section{Reviewer 2}

\begin{questions}
\question one
\begin{solution}
 We opted to follow the ABC style of ...
\end{solution}
\question  two
\begin{solution}
Section 2 has been updated and the discussion ...
\end{solution}
\end{questions}


\section{Reviewer 3}

\begin{questions}
\question one
\begin{solution}
 We opted to follow the ABC style of ...
\end{solution}
\question  two
\begin{solution}
Section 2 has been updated and the discussion ...
\end{solution}
\end{questions}


\end{document}
