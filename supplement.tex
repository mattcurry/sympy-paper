%% Submissions for peer-review must enable line-numbering
%% using the lineno option in the \documentclass command.
%%
%% Preprints and camera-ready submissions do not need
%% line numbers, and should have this option removed.
%%
%% Please note that the line numbering option requires
%% version 1.1 or newer of the wlpeerj.cls file.

\documentclass[fleqn,10pt,lineno,numbers]{wlpeerj} % for journal submissions
% \documentclass[fleqn,10pt]{wlpeerj} % for preprint submissions

%\usepackage[numbers]{natbib}

\usepackage{lmodern}
\usepackage[T1]{fontenc}
\usepackage[utf8]{inputenc}
\usepackage[scaled=0.8]{DejaVuSansMono}

\usepackage{hyperref}
\usepackage{graphicx}
\usepackage[all]{xy}
\usepackage{amsmath}
\usepackage{caption}
\graphicspath{ {images/} }

% Makes quote characters in monospace font not be curly
\usepackage{upquote}

\usepackage{amsmath}
\usepackage{url}
\usepackage{hyperref}

% this is required for all the \url{} commands in the bib file
%\usepackage{hyperref}

% for nice units
\usepackage{siunitx}

% for images: png, pdf, etc
\usepackage{graphicx}

% for nice table formatting, i.e., /toprule, /midrule, etc
\usepackage{booktabs}

% to allow for \verb++ declarations in captions.
\usepackage{cprotect}

% to allow usage of \mathbb symbols
\usepackage{amssymb}

\usepackage{longtable}

\usepackage{listings}


\begin{document}

\flushbottom
\thispagestyle{empty}%
\vskip-36pt%
{\raggedright\sffamily\bfseries\fontsize{20}{25}\selectfont SymPy: Symbolic Computing in Python\par}%
\vskip10pt
{\raggedright\sffamily\fontsize{12}{16}\selectfont  Supplementary material\par}
\vskip25pt%

The supplementary material take a deeper look at certain topics in SymPy for
which there was not enough room to discuss in the paper.
Section \ref{suppsec:Gruntz} discusses the Gruntz algorithm, used to
calculate limits in the SymPy.  Sections \ref{suppsec:Series}--\ref{suppsec:Ten}
discuss in depth some selected submodules.  Section \ref{suppsec:numsimpl}
discusses numerical simplification.  Section \ref{suppsec:examples} provides
additional examples for topics, discussed in the main paper.  In sections
\ref{sympy-gamma} and \ref{sympy-live} the SymPy Gamma
and the SymPy Live projects are introduced.  Finally, section \ref{comp-mma}
has a
brief comparison of SymPy with Wolfram Mathematica, and
section \ref{other-proj} lists some projects that depend on SymPy.

As in the paper, all examples in the supplement assume that the following
has been run:
\begin{verbatim}
>>> from sympy import *
>>> x, y, z = symbols('x y z')
\end{verbatim}


\section{Limits: The Gruntz Algorithm}
\label{suppsec:Gruntz}
\input{gruntz.tex}

% Series module (Formal Power Series, Fourier Series)
\section{Series}
\label{suppsec:Series}
\input{series.tex}

% Logic module
\section{Logic}
\label{suppsec:Logic}
\input{logic}

\section{Diophantine Equations}
\label{suppsec:Dioph}
\input{diophantine}

% Sets
\section{Sets}
\label{suppsec:Sets}
\input{sets}

\section{Statistics}
\label{suppsec:Stats}
\input{stats}

\section{Category Theory}
\label{suppsec:Cat}
\input{categories}

\section{Tensors}
\label{suppsec:Ten}
\input{tensors}

\section{Numerical Simplification}
\label{suppsec:numsimpl}
\input{nsimplify}

\section{Examples}
\label{suppsec:examples}
\input{examples}

\section{SymPy Gamma}\label{sympy-gamma}

\input{gamma}

\section{Comparison with Mathematica}
\label{comp-mma}
\input{comparison_with_mma}

\bibliography{paper}

\end{document}
